\chapter{Storage?}

\section{Safe Storage}
%Why? How? +(keep all the entries)/-(performance, delays) -> keep the possibility to run 
%usual wrappers

\todo{TODO configure the listings style}


\begin{rubycode}
def mymethod 
	@foo.each { |bar| bar.to_s }
end
\end{rubycode}

\begin{javacode}
public class Test {

	private int test;

	public Test () {}

	public Test (int level) {
		this.test = test;
	}
}
\end{javacode}

\begin{xmlcode}
<tag attribute=''val1''>
	<tag2 p=''5'' />
</tag>
\end{xmlcode}

\begin{htmlcode}
<html>
	<head>
		<title>Test Page</title>
	</head>
	<body>
		<p>Test</p>
	</body>
</html>
\end{htmlcode}



diagram (two processes)...

\subsection{Use Cases}
%if processes stop or communication breaks
example with a concrete run, ss\_mem\_wrapper

\subsection{Software Architecture}

Abstract (SafeStorage wrapper or not) for GSN

List of defaults parameters

Configuration in the build.xml file

\subsection{Safe Storage modes}

\begin{verbatim}
ant start-acquisition
ant clean-acquisition
ant stop-acquisition
\end{verbatim}

\subsection{How to develop a Wrapper}
