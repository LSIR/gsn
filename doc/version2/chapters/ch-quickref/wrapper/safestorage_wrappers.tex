\subsection{Safe Storage Wrappers Default parameters}

The parameters shown on the \listingref{table:safe_storage_parameters} must be added to each \wrapper that support the 
Safe Storage feature. An example of \wrapper that use this feature is described in \chapref{tinyos-mig:wrapper}.

\begin{table*}[!htp]
	\centering
	{\normalfont\footnotesize
	\begin{tabulary}{\textwidth}{|C|C|C|C|J|}%
	\hline
		\multicolumn{5}{|c|}{\textbf{Safe Storage Parameters}} \\
	\hline
	\hline
		\textbf{Parameter Name} &
		\textbf{Type} &
		\textbf{Mandatory} &
		\textbf{Default} &
		\textbf{Description} \\
	\hline
	\hline
		ss-host &
		String &	
		Yes &
		None &
		The machine host name that runs the Safe Storage \\
	\hline
		ss-port &
		Integer &	
		Yes &
		None &
		The server port on which Safe Storage listen for connections \\
	\hline
		wrapper-name &
		String &	
		Yes &
		None &
		The Safe Storage side wrapper full classname (must extends \inlinecode{gsn.acquisition2.wrappers.AbstractWrapper2}) 
		or the short name from the \path{conf/safe\_storage\_wrappers.properties} file. \\
	\hline
		wrapper-keep-processed-ss-entries &
		Boolean &
		No &
		true &
		If this option is set to \inlinecode{true}, all the entries (processed or not) kept into the Safe Storage storage.
		If this option is set to \inlinecode{false}, the processed entries are removed from the Safe Storage storage once processed. \\
	\hline
		continue-on-error &
		Boolean &	
		No &
		true &
		Not yet implemented \\
	\hline
	\end{tabulary}
	}
	\caption{Safe Storage Parameters}
	\label{table:safe_storage_parameters}
\end{table*}