\graphicspath{{chapters/ch-architecture/figures/}}

\chapter{GSN Architecture}

\section{Data Acquisition}

 Before filtering and processing data, GSN needs to receive it. GSN considers two types of data sources: event-based and polling-based.
 In the first case, data is sent by the source and a GSN method is called when it arrives.
 Serial ports, network (TCP or UDP) connections, wireless webcams fall in this case. In the latter one, GSN periodically asks the 
 source for new data. This is the case of an RSS feed or a POP3 email account.

\section{Safe Storage}

%Why? How? +(keep all the entries)/-(performance, delays) -> keep the possibility to run 
%usual wrappers



diagram (two processes)...
\image{safestorage.pdf}{0.8}{0}{Safe Storage Software Architecture}{image:ss_software_architecture}

\subsection{Use Cases}
%if processes stop or communication breaks
example with a concrete run, ss\_mem\_wrapper

\subsection{Software Architecture}

Abstract (SafeStorage wrapper or not) for GSN

List of defaults parameters

Configuration in the build.xml file

\subsection{Safe Storage modes}

\begin{bashcode}
ant start-acquisition
ant clean-acquisition
ant stop-acquisition
\end{bashcode}

\subsection{How to develop a Safe Storage Wrapper}




































\newpage
\subsection{Examples}
\begin{rubycode}[caption={Ruby Code Example}, label=listing:ruby:example]
# A method
def mymethod 
	@foo.each { |bar| bar.to_s }
	@@foobar = "a string"
end
\end{rubycode}

\begin{javacode}
/**
 * Java Class
 */
public class Test {
	private int test;
	public Test () {}
	public Test (int level) {
		// A comment
		String b = "my test string";
		this.test = test;
	}
}
\end{javacode}

\begin{xmlcode}[caption={XML Code Example}, label=listing:xml:example]
<!-- one comment -->
<tag attribute="val1">
	<tag2 p="5" />
</tag>
\end{xmlcode}

\begin{htmlcode}
<!-- one comment -->
<html>
	<head>
		<title>Test Page</title>
	</head>
	<body>
		<p style="display: none;">Test</p>
	</body>
</html>
\end{htmlcode}

\cite{dummyarticle}

