\graphicspath{{chapters/ch-developer-guide/figures/}}

\section{Writing a Virtual Processing Class \label{sec:vsp}}

\subsection{The AbstractVirualSensor class}

All virtual sensors are subclass of the AbstractVirtualSensor (package gsn.vsensor).
It requires its subclasses to implement the following three methods:
\begin{itemize}
 \item \inlinecode{public boolean initialize()}\\
 \item \inlinecode{public void dataAvailable(String inputStreamName, StreamElement se)}\\
 \item \inlinecode{public void finalize()}\\
\end{itemize}
\inlinecode{initialize()} is the first method to be called after object creation. It should
configure the virtual sensor according to its parameters, if any, and return true
in case of success, false if otherwise. If this method returns false, GSN will
generate an error message in the logs and stops using the virtual sensor.

\inlinecode{finalize()} is called when GSN destroys the virtual sensor. It should release
all system resources in use by this virtual sensor. This method is typically called
when we want to shutdown the GSN instance.

\inlinecode{dataAvailable} is called each time that GSN has data for this virtual sensor,
according to its configuration. If the virtual sensor produces data, it should
encapsulate this data in a \inlinecode{StreamElement} object and deliver it to GSN by
calling \inlinecode{dataProduced(StreamElement se)}.

Note that a Virtual Sensor should always use the same StreamElement struc-
ture for producing its data. Changing the structure type is not allowed and
trying to do so will result in an error. However, a virtual sensor can be con-
figured at initialization time what kind of StreamElement it will produce. This
allows to produce different types of StreamElement by the same VS depending
on its usage. But one instance of the VS will still be limited to produce the
same structure type. If a virtual sensor really needs to produce several different
stream elements, user must provide the set of all possibily fields in the stream
elements and provide Null whenever the data item is not applicable.

\subsection{Reading Initialisation Parameters}


As you noticed in the Chart virtual sensor, a virtual sensor can receive param-
eters from the virtual sensor description file and use them in the initialization
process. An initialization parameter has a name and a value both are in the
form of String. The parameter name is cases insensitive.
For example, in order to read the value of the ``my-parameter'' parameter
form the following code snippets

\begin{xmlcode}
...
<processing-class>
<class-name>gsn.vsensor.ChartVirtualSensor</class-name>
<init-params>
<param name="my-parameter">123456</param>
</init-params>
...
You can use the following java code in the initialize method :
TreeMap < String , String > params = getVirtualSensorConfiguration(
).getMainClassInitialParams( );
String value = params.get("my-parameter");
if (value ==null){ // parameter is missing.
// use default value or return false with an error message.
}
\end{xmlcode}

\subsection{The StreamElement class}
A StreamElement is a GSN class that encapsulates data. It has a data types
structure (a DataField array), a data values structure (a Serializable array) and
a timestamp.

\subsection{Writing your own graphical user interface}
A virtual sensor is not limited to raw data processing. You can call any other
Java library, including Swing classes. An introduction to GUI programming is
outside the scope of this document. You can have a look at the HCIProtocol-
GUIVS class to see how such an interface can be implemented.
A simple way to go is to create the graphical components (like a JFrame)
in the \inlinecode{initialize()} method and at the same time define the events logic
\inlinecode{(eventListeners...)}. In the \inlinecode{dataAvailable()} method, received data can be
sent to graphical components to present the information to the user. Beware
that there may be concurrency problems since your GUI is running with the
Swing event thread while your virtual sensor is run by a GSN thread.


\subsection{Feedback channel to a Virtual Sensor}

\inlinecode{public boolean dataFromWeb ( String action,String[] paramNames, Serializable[] paramValues )}