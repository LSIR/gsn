\subsection{ss\_tinyos-mig \wrapper \label{tinyos-mig:wrapper}}

The TinyOS \wrapper can receive data from both version 1.x and 2.x TinyOS based networks.
This \wrapper can interract with any TinyOS compatible Base Station and any type of TinyOS Packet Source.
For instance The TinyOS wrapper can interact with the serial forwarder (provided by TinyOS distribution) which 
inturn presents a sensor network. In order to use the TinyOS wrapper with a sensor network you need to first generate the java
representation of the message structures used in the network (in TinyOS they are defined in the \inlinecode{.h} files). 
NesC language provides a tool called \inlinecode{mig} (message
interface generator for nesC) for this purpose. The tool has a standard man page
documentation in addition to being described in the lession 4 of the TinyOS 2
tutorial
\footnote{Further informations can be found at\\ \url{http://www.tinyos.net/tinyos-1.x/doc/nesc/mig.html}\\ \url{http://www.tinyos.net/tinyos-2.x/doc/html/tutorial/lesson4.html}\\ \url{http://www.tinyos.net/tinyos-1.x/doc/tutorial/lesson6.html}}.

Both TinyOS 1.x and 2.x are using very same package structures and class names which generates conflict when 
you have both TinyOS1.x and 2.x jar files in your classpath (which is the case for \gsn).
To solve this issue, \gsn provides a slightly modified version of the TinyOS 1.x java tools by renaming the \inlinecode{net.tinyos.xxx} package to \inlinecode{net.tinyos1x.xxx}.
Thus the messages classes generated with MIG must be configured depending on the version of TinyOS your are using. The \tableref{table:tinyos_mig_version} shows the class that your
MIG generated messages must extend (directly or not directly).

\begin{table*}[!htp]
	\centering
	{\normalfont\footnotesize
	\begin{tabulary}{\textwidth}{|C|J|}%
	\hline
		\textbf{TinyOS version} &
		\textbf{Must extends} \\
	\hline
	\hline
		1.x &
		\inlinecode{net.tinyos1x.message.Message} \\
	\hline
		2.x &
		\inlinecode{net.tinyos.message.Message} \\
	\hline
	\end{tabulary}
	}
	\caption{Superclass for TinyOS messages classes}
	\label{table:tinyos_mig_version}
\end{table*}

\begin{table*}[!htp]
	\centering
	{\normalfont\footnotesize
	\begin{tabulary}{\textwidth}{|C|C|C|C|J|}%
	\hline
		\multicolumn{5}{|c|}{\textbf{ss\_tinyos-mig \wrapper Parameters}} \\
	\hline
	\hline
		\textbf{Parameter Name} &
		\textbf{Type} &
		\textbf{Mandatory} &
		\textbf{Default} &
		\textbf{Description} \\
	\hline
	\hline
		source &
		String &	
		Yes &
		None &
		The TinyOS data source \newline{}eg. \url{sf@serial.forwarder.url:9001}  \\
	\hline
		message-classname &
		String &
		Yes &
		None &
		The full package path to the message class generated by MIG\newline{}eg. \inlinecode{ch.ethz.permafrozer.DozerDataMsg} \\
	\hline 
		message-length &
		Integer &
		No &
		\inlinecode{DEFAULT\_MESSAGE\_SIZE} &
		Override the default size of the messages received \\
	\hline
		getter-prefix &
		String &
		No &
		get\_ &
		The methods of the message class that contain this prefix will be added in the output structure. 
		Keep the default value for the messages generated with MIG. \\
	\hline
	\hline
		\multicolumn{2}{|l}{Support Safe Storage} &
		\multicolumn{3}{l|}{Yes (Parameters listed on \tableref{table:safe_storage_parameters} must be added.)} \\
		\multicolumn{2}{|l}{SS \wrapper Classname} &
		\multicolumn{3}{l|}{gsn.acquisition2.wrappers.MigMessageWrapper2 (mig2)} \\
		\multicolumn{2}{|l}{GSN \wrapper Classname} &
		\multicolumn{3}{l|}{gsn.acquisition2.wrappers.MigMessageWrapperProcessor (ss\_tinyos-mig)} \\

	\hline
	\end{tabulary}
	}
	\caption{ss\_tinyos-mig \wrapper Parameters}
	\label{table:tinyos-mig_parameters}
\end{table*}

This \wrapper infers the output structure from the methods names contained into the MIG generated message class and the superclasses. To filter these methods, 
This \wrapper applies a prefix pattern matching on the methods names. The mapping between the TinyOS types and the types to use in your \vsd are shown on the 
\tableref{table:tinyos-mig_wrapper_output_structure}.

\begin{table*}[!htp]
	\centering
	{\normalfont\footnotesize
	\begin{tabulary}{\textwidth}{|C|C|C|J|}%
	\hline
		\multicolumn{4}{|c|}{\textbf{ss\_tinyos-mig \wrapper Output Structure}} \\
	\hline
	\hline
		\textbf{nesC} &
		\textbf{Java} &
		\textbf{\vsd} &
		\textbf{Description} \\
	\hline
	\hline
		nx\_int8\_t &
		byte &
		TINYINT &
		8-bit signed \\
	\hline 
		nx\_uint8\_t &
		short &
		SMALLINT &
		8-bit unsigned \\
	\hline 
		nx\_int16\_t &
		short &
		SMALLINT &
		16-bit signed \\
	\hline 
		nx\_uint16\_t &
		int &
		INTEGER &
		16-bit unsigned \\
	\hline 
		nx\_int32\_t &
		int &
		INTEGER &
		32-bit signed \\
	\hline 
		nx\_uint32\_t &
		long &
		BIGINT &
		32-bit unsigned \\
	\hline 
		NOT SUPPORTED &
		float &
		DOUBLE &
		32-bit floating point number \\
	\hline 
		NOT SUPPORTED &
		double &
		DOUBLE &
		64-bit floating point number \\
	\hline 
	\hline 
		\multicolumn{4}{|c|}{Arrays of the listed types are also supported.} \\
	\hline 
	\end{tabulary}
	}
	\caption{ss\_tinyos-mig \wrapper Output Structure}
	\label{table:tinyos-mig_wrapper_output_structure}
\end{table*}
